% File: main.tex
\documentclass[15pt]{extarticle}

% --- Minimal, Overleaf-ready setup ---
\usepackage[margin=1in]{geometry}
\usepackage{times}
\usepackage{url}
\usepackage{graphicx}
\usepackage{enumitem}
\usepackage{natbib}
\usepackage{titlesec}
\usepackage{setspace}
\setlist{nosep,leftmargin=*,itemsep=2pt,topsep=2pt}
\setlength{\parskip}{4pt}
\setlength{\parindent}{0pt}
\titlespacing*{\section}{0pt}{6pt}{4pt}
\titlespacing*{\subsection}{0pt}{5pt}{3pt}

\title{\textbf{Project Proposal (Max 2 Pages, 15 pt)}\\
\large Reproduction Plan: \emph{[Insert Paper Title Here]}}
\author{[Anonymous / Your Name]}
\date{}

\begin{document}
\maketitle

% Optional: keep to 1–3 sentences
\textbf{Abstract.} One–to–three sentence overview of your reproduction plan.

\section*{Introduction (3)}
[Motivate the problem, why reproduction matters, your goals.]
\textbf{Initial LLM prompt \& validation:}
\begin{itemize}
  \item \emph{Prompt used:} [1–2 lines]
  \item \emph{Initial LLM output (abridged):} [key points]
  \item \emph{Validation:} Correctness [low/med/high]; Relevance [low/med/high]; Helpfulness [low/med/high]
  \item \# of prompts: [N]. If the initial prompt failed: [why/what you changed]
\end{itemize}

\section*{Problem Statement (2)}
\textbf{What is the problem statement of the paper you are proposing to reproduce?}\\
[1–2 precise sentences.]
\textbf{Initial LLM prompt \& validation:}
\begin{itemize}
  \item \emph{Prompt used:} [...]
  \item \emph{Initial LLM output:} [...]
  \item \emph{Validation:} [correct/relevant/helpful?] \quad \#prompts: [N]
  \item If initial prompt didn’t work: [what was wrong]
\end{itemize}

\section*{Citation to the Original Paper (1)}
[Full citation with URL/DOI.]

\section*{Methodology (6)}
\subsection*{Specific Approach (2)}
\textit{Models:} [Teacher / Student]\\
\textit{Techniques:} [KD variant(s), uncertainty, contrastive, aggregation]\\
\textit{Evaluation metrics:} [Accuracy, Specificity, Sensitivity, AUC, …]
\\[2pt]
\textbf{Initial LLM prompt \& validation:}
\begin{itemize}
  \item \emph{Prompt used:} [...]
  \item \emph{Initial LLM output:} [...]
  \item \emph{Validation:} [correct/relevant/helpful] \quad \#prompts: [N]
  \item If initial prompt didn’t work: [what was wrong]
\end{itemize}

\subsection*{Novelty / Relevance / Hypotheses (2)}
[What’s novel/relevant? Why better than baselines? Explicit hypotheses H1, H2, …]
\\[2pt]
\textbf{Initial LLM prompt \& validation:}
\begin{itemize}
  \item \emph{Prompt used:} [...]
  \item \emph{Initial LLM output:} [...]
  \item \emph{Validation:} [correct/relevant/helpful] \quad \#prompts: [N]
  \item If initial prompt didn’t work: [what was wrong]
\end{itemize}

\subsection*{Ablations / Extensions Planned (2)}
[Bullets: what you’ll ablate/extend; is the hypothesis legitimate and how you’ll test it.]

\section*{Data Access and Implementation Details (5)}
\subsection*{Description of Data/Model Access (2)}
\textbf{What is the dataset they use and do they use a public codebase?}\\
[Dataset name(s), access path (public/credentialed), any repo/demo, your access plan.]
\\[2pt]
\textbf{Initial LLM prompt \& validation:}
\begin{itemize}
  \item \emph{Prompt used:} [...]
  \item \emph{Initial LLM output:} [...]
  \item \emph{Validation:} [correct/relevant/helpful] \quad \#prompts: [N]
  \item If initial prompt didn’t work: [what was wrong]
\end{itemize}

\subsection*{Discussion of Feasibility (2)}
[Compute (GPUs/CPUs/RAM), time, storage, libs, risks \& mitigations.]

\subsection*{Use Existing Code? (1)}
[Explicit: use / partial reuse / reimplement, with justification.]

\subsection*{Two-Page Limit (1)}
[Affirm: This document is $\leq$ 2 pages in required format.]

\section*{Additional Comments on Using LLM Prompts}
[How you tweaked prompts for usefulness: structure, constraints, citations, bullets, etc.]

% References / Appendix (Unlimited)
\bibliographystyle{plainnat}
\bibliography{refs} % Create refs.bib or replace with manual entries.

\end{document}
